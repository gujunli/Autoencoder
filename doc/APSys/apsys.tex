% Use standard two-column Latex article class with 
% smaller margins and blank space for copyright notice

\documentclass[twocolumn,notitlepage,11pt,letter]{article}

% For pdflatex
\setlength{\pdfpagewidth}{8.5in}
\setlength{\pdfpageheight}{11in}

% Default margins are too big
% Magic below sets it to 1 inch all around
\setlength\oddsidemargin{+0.075in}
\setlength\evensidemargin{+0.075in}
\setlength\textwidth{6.5in}
\setlength\topmargin{-0.5in}
\setlength\textheight{8.5in}
\setlength\columnsep{0.33in}

% Blank space for ACM to add copyright notice to .pdf file
\usepackage{float}
\def\acmblank{\begin{figure}[b]\vspace{1in}\end{figure}}


\begin{document}
\title{DNN paper outline}
\author{Marco Maohua Zhu}
\date{March 6, 2014}

\maketitle
\acmblank{}



\section{Introduction}

Stochastic gradient descent is a widely used method for deep neural network training. \cite{QuocLe2011} \\
Left blank because this is an outline.

\section{Background}

DNN is hot in machine learning research. SGD is the most common training algorithm but it is difficult to tune and parallelize. CG and L-BFGS, though more complicated, now can be used thanks to the availability of high performance multi-core CPUs, GPUs and computer clusters.\\
Challenge: too many network parameters, hard to compute.\\
We need an optimized implementation on suitable (heterogeneous) platforms.

\section{Motivation}

There are several available architectures to build a heterogeneous system for deep learning.\\
It is not trivial to map deep learning algorithms on heterogeneous systems.\\
Researchers and companies need an analysis of advantages and bottlenecks of each platform given the deep learning task.

\section{Mainstream architectures}

An introduction to multi-core CPU(SIMD in CPU, SSE/AVX), APU and discrete GPU architecture.

\section{Implementation and optimization}

How we map the algorithm to APU and discrete GPU.

\section{Evaluation results and analysis}

Performance, power and performance per watt. Present bottlenecks of each platform.

\section{Further analysis}

If there is an alternative choice which seems better, do an experiment.

\section{Conclusion}

Insights.


\bibliographystyle{plain}
\bibliography{references}
\end{document}
